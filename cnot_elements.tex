\documentclass[11pt]{article}
\usepackage[top=2.5cm,bottom=2.5cm,right=2.5cm,left=2.5cm]{geometry}
\usepackage{graphicx,float}
\def\C{\rm CNOT}
\title{CNOT scenarios}
\date{}
\begin{document}
\maketitle

The CNOT gates are indexed as follows.
%\begin{table}[H]
%\begin{center}
\begin{tabular}{|c|c|}
\hline
Index & Gate \\
\hline
12 & $\C_{12}$ \\
13 & $\C_{13}$ \\
14 & $\C_{21}$ \\
15 & $\C_{23}$ \\
16 & $\C_{31}$ \\
17 & $\C_{32}$ \\
\hline
\end{tabular}
%\end{center}
%\end{table}

%\vspace{1cm}

\begin{figure}[H]
\begin{center}
\includegraphics[scale=0.5]{case1.jpg}	\includegraphics[scale=0.5]{case2.jpg}

\includegraphics[scale=0.5]{case3.jpg}	\includegraphics[scale=0.5]{case4.jpg}

\vspace{0.5cm}

\includegraphics[scale=0.5]{case5.jpg}	\includegraphics[scale=0.5]{case6.jpg}

\vspace{0.5cm}

\includegraphics[scale=0.5]{case7.jpg}	\includegraphics[scale=0.5]{case8.jpg}
\caption{For each circuit, the list $C = [\ldots]$ denotes the indices of the gates.}
\label{fit:cnots}
\end{center}
\end{figure}

\end{document}  